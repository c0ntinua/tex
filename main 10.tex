\documentclass{article}
\usepackage[utf8]{inputenc}
\usepackage[margin=0.5in]{geometry}
\usepackage{graphics}
\setlength{\parindent}{0pt}

\title{Practice Final Exam (Final Homework)}

\begin{document}

\maketitle

\section{General Notes}
To save paper, I've left just enough space between problems for you to write the answer down. On test day, you'll use separate sheets of paper to record your work on a test sheet extremely similar to this practice test sheet. This way you won't run out of space (use as many pages as you like, and turn them in stapled with work for each problem circled and labelled.) \\

Please note that showing work is necessary. You don't have to show every tiny step, but you have to demonstrate mastery of a technique. This also allows me to give partial credit even when the answer is wrong. Because this practice test is so similar to the actual final, an excellent performance on the final is a fair expectation, so please don't rely on partial credit as a substitute for practicing until you can reliably do the whole problem correctly (the last step is checking your answer when possible.)


\section{Polynomial Multiplication and Division}

(1) Evaluate the product $[(3p-2)+ 5q][(3p-2) -5q]$.

[10 points] [Section R.3, Example 9A, page 30]

Remember that this product is of the form $(x + y)(x -y)$, so that you can use a formula to make this multiplication easier (study the example solution, which we'll also look at in class.) It is acceptable if you prefer to just multiply every term in the first set of brackets by every term in the second set of brackets (which is the more general method.) The main thing is to show your work and get the correct answer. We will talk about how to partially check this kind of problem (it's not as easy as I'd like, but some checks are possible.)\\\\\\\\\\\




(2) Divide $4m^3-8m^2+5m + 6$ by $2m-1$. 


[10 points] [Section R.3, Example 10, page 31]

It's acceptable to use synthetic division on this problem, but we will not go over synthetic division in the review. I'll use old-fashioned polynomial long division in class. This is another problem that's not easy to check, but we will talk about a method that can make you more confident about your answer.\\\\\\\\\\\

(3) Divide $3x^3-2x^2-150$ by $x^2-4$. 


[10 points] [Section R.3, Example 11, page 31]

Be on the lookout and make adjustments for the missing terms. It's acceptable to use synthetic division on this problem, but we will not go over synthetic division in the review. This is another problem that's not easy to check, but we will talk about a method that can make you more confident about your answer.\\\\\\\\\\\\\\\

\section{Linear Equations}
In general, solutions to equations (including linear equations) can be checked easily by plugging the proposed solution back in and making sure that the mathematical statement is true. Answer-checking is simply part of the problem-solving process. This means that less partial credit will be given for wrong answers that were not checked. It's possible that you'll discover a wrong answer during a check and not have time or be able to get the right answer. In that case, you will still get credit for knowing what/that you don't know. \\\\

(4) Solve the equation $3(2x-4) = 7 - (x + 5)$.

[10 points] [Section 1.1, Example 1, page 89]

This should be one of the easier problem types on the test. It's also straightforward to check your answer.\\\\\\\\\\\\\\\

(5) Solve the equation $\frac{2x+4}{3} + \frac{1}{2}x = \frac{1}{4}x-\frac{7}{3}$.

[10 points] [Section 1.1, Example 1, page 89]

This one is little trickier, since you'll need to use your skill with fractions.  Fortunately it's still straightforward to check your answer.\\\\\\\\\\\\\\\

(6)  A chemist needs a 20 percent solution of alcohol. She has a 15 percent solution and a 30 percent solution on hand (as much as she might need of either). How many liters of the 15 percent solution should she add to 3 liters of the 30 percent solution in order to get a 20 percent solution?

[10 points] [Section 1.3, Example 3, page 97]

The math for this problem is fairly simple, but (as with most story problems) setting up the math can be challenging. You will be given this exact same problem type on the test (a mixture problem specifically and not just a story problem involving a linear equation.)\\\\\\\\\\\\\\\

\section{Quadratic Equation}
We'll close out the test with 4 quadratic equations. I recommend sticking with the one-size-fits-all quadratic formula method, but it's OK if you use other methods, as long as you show your work and get the right answer. You should also check for answers, which is pretty easy with quadratic equations. Also, don't forget the small, extra step that is sometimes necessary of putting the equation in standard form so that $0$ is on one side by itself. Then you can get $a,b$ and $c$ as needed for the quadratic formula.\\\\

(7) Solve the equation $x^2 -4x  = -2$.

[10 points] [Section 1.4, Example 5, page 117]\\\\\\\\\\\\\\\


(8) Solve the equation $2x^2 = x - 4$.

[10 points] [Section 1.4, Example 6, page 118]\\\\\\\\\\\\\\\





(9) Solve the equation $x^2 -6x = -7$.

[10 points] [Section 1.4, Problem 53, page 122]\\\\\\\\\\\\\\\



(10) Solve the equation $\frac{1}{2}x^2 +\frac{1}{4}x - 3 = 0$.

[10 points] [Section 1.4, Problem 59, page 123]

This one is a little trickier, since you'll need to handle fractions while also using one of the techniques for solving quadratic equations (I recommend mastering the formula approach, which works in every situation.)\\\\\\\\\\\\\\\








\end{document}
