\documentclass{article}
\usepackage[utf8]{inputenc}

\documentclass{article}
\usepackage[utf8]{inputenc}
\usepackage[usenames]{color} %used for font color
\usepackage{amssymb} %maths
\usepackage{amsmath} %maths
\usepackage{setspace}
\usepackage{ mathrsfs }
\usepackage[margin=0.5in]{geometry}
\usepackage{graphics}
\setlength{\parindent}{0pt}
\newcommand{\odds}{ \mathscr{O}}
\newcommand{\evens}{ \mathscr{E}}

\newcommand{\rationals}{ \mathscr{Q}}
\newcommand{\hyperset}{ \mathscr{H}}
\newcommand{\targetset}{ \mathscr{R}}
\newcommand{\sourceset}{ \mathscr{D}}
\newcommand{\reals}{ \mathbb{\elt}}

\newcommand{\disjoints}{ X}
\newcommand{\nondisjoints}{E}
\newcommand{\naturals}{ \mathscr{N}}
\newcommand{\carpenter}{ \xi }
\newcommand{\leaper}{l}
\newcommand{\climber}{ f}
\newcommand{\homo}{\theta}
\newcommand{\elt}{a}
\newcommand{\homoelt}{b}
\newcommand{\altelt}{s}
\newcommand{\rooter}{\Psi}
\newcommand{\invmark}{\overline}
\newcommand{\odds}{ \mathscr{O}}
\newcommand{\sequencer}{\sigma}
\newcommand{\umbrella}{\beta}
\newcommand{\prevert}{\zeta}
\newcommand{\targetname}{bundle }
\newcommand{\targetnameplural}{bundles }
\newcommand{\intersector}{\Xi}
\newcommand{\interceptor}{\Omega}
\newcommand{\wed}{\alpha}
\newcommand{\Segmenter}{\Upsilon}
\newcommand{\ahyperset}{H}
\newcommand{\Homo}{\Phi}


\title{\textbf{Dyadic Rods}}
\author{S. Z. }


\begin{document}

\maketitle

\section{Introduction}
While Dedekind identified a real number with every possible separation of the rationals into a left part and right part, the system sketched below identifies a real number with every possible initial stretch of the positive dyadic numbers. \\

\section{Preliminaries}


\textbf{Definition}\\
Let $\naturals $ be the set of positive integers.\\

\textbf{Definition}\\
Let $\odds = 2\naturals - 1 $ be the set of positive odd integers.\\

\textbf{Definition}\\
For $n \in \naturals$, define  $\nondisjoints_n = \{ 2^{-n}j : j \in \naturals \}$. \\

\textbf{Proposition}\\
$\forall n \in \naturals \;\; \nondisjoints_n \subset \nondisjoints_{n+1}.$\\

(1) $\elt \in \nondisjoints_n \implies \elt = 2^{-n}j = 2^{-(n+1)}2j \in \nondisjoints_{n+1}$.\\

\textbf{Proposition}\\
$\forall n \in \naturals \;\; \forall k \in \naturals \;\; \nondisjoints_n \subset \nondisjoints_{n+k}.$\\

\textbf{Definition}\\
For  $n \in \naturals$, define $\disjoints_n = \{2^{-n} j : j \in \odds \}$.\\

\textbf{Definition}\\
Also define $\disjoints_0 = \naturals$.\\

\textbf{Proposition}\\
$m \ne n \implies \disjoints_m \cap \disjoints_n = \emptyset.$\\

\textbf{Proposition}\\
$\bigcup_{n = 1}^\infty \nondisjoints_n =  \bigsqcup_{n= 0}^\infty \disjoints_n$.\\

\textbf{Definition}\\
Define $\sourceset = \bigcup_{n = 1}^\infty \nondisjoints_n$.\\

\textbf{Proposition}\\
$x,y \in \sourceset \implies  x+y,xy \in \sourceset$.\\


\textbf{Definition}\\
For $A,B \subset \sourceset$, define $A < B$ to mean $A \subsetneqq B $.\\

\section{Dyadic Rods}
\textbf{Definition}\\
Define $\targetset = \{A \subset \sourceset : \;\; \emptyset < A < \sourceset, \;\; a' < a \in A \implies a' \in A,\;\; \forall a \in A \;\; \exists a'  \in A \;\; a < a'\;\; \}$.\\

\textbf{Definition}\\
Define $\homo : \sourceset \to \targetset$ by $\homo(\altelt)  = \{ \elt \in \sourceset : \elt < \altelt \}$.\\

\textbf{Proposition}\\
$\forall \homoelt \in \sourceset \;\; \homo(\homoelt) \in \targetset.$\\

(1) $0 < 2^{-1}\homoelt < \homoelt \implies 2^{-1}\homoelt \in \homo(\homoelt)$.\\ 
(2) $\elt \in \homo(\altelt) \implies \elt < \altelt \implies \altelt \notin\homo(\homoelt)$.\\
(3) $\elt < \altelt \in \homo(\altelt) \implies 0 < \elt < \altelt < \homoelt \implies \elt \in \homo(\homoelt).$\\
(4) $\elt \in \homo(\homoelt) \implies \elt < \homoelt \implies \elt < 2^{-1}(\elt+ \homoelt) < \homoelt \implies 2^{-1}(\elt+ \homoelt) \in \homo(\homoelt)$.\\




\section{An Enumerating Sequence}\\

\textbf{Definition}\\
For $A \in \targetset$ and $ n \in \naturals$, define $\intersector^A_n = A \cap \disjoints_n$.\\

\textbf{Proposition}\\
$A = \bigsqcup_n^\infty \intersector^A_n$.\\

\textbf{Definition}\\
Define $\wed^A = \min \{ m \in \naturals : 2^{-m} \in A \}$. \\

\textbf{Definition}\\
For $A \subset \sourceset$, define $|A|$ to be the number of elements of $A$, where $\infty$ represents countable infinity.\\ 

\textbf{Proposition}\\
$\forall n \in \naturals \;\; n \ge \wed^A \implies 0 <  |\intersector^A_n| < \infty$.\\

\textbf{Definition}\\
Let $\elt \in \interceptor_i,  \elt' \in \interceptor_j$. Define $\elt \prec \elt'$ if either $i < j$ or $i = j$ and $\elt < \elt'$.\\

\textbf{Proposition}\\
A is well-ordered by $\prec.$\\

\textbf{Note}\\
Each nonempty subset of a well-ordered set has a minimum.\\ 

\textbf{Definition}\\
For $A \in \targetset$, define $\carpenter^A : \naturals \to A$ so that $\carpenter^A_n$ is the $n$th element of $A$ according to $\prec$.\\


\textbf{Example}\\
Let $A = \{ \elt \in \sourceset : \elt < 1 \}.$ Then $\carpenter^A = \frac{1}{2},\frac{1}{4},\frac{3}{4},\frac{1}{8},\frac{3}{8},\frac{5}{8}, ... $ and $\carpenter^A_2 = \frac{1}{4}$.\\

\textbf{Note}\\
Each element of $A$ appears exactly once as a term in the sequence $\carpenter_A$.\\


\section{A Nondecreasing Sequence}


\textbf{Definition}\\
For $A \in \targetset$ and $ n \in \naturals$, define $\interceptor ^A_n = A \cap \nondisjoints_{n + \wed^A - 1}$.\\

\textbf{Proposition}\\
$\forall n \in \naturals \;\; |\interceptor^A_n| \in \naturals$.\\

\textbf{Proposition}\\
$A = \bigcup_n^\infty \interceptor^A_n$.\\


\textbf{Definition}\\
For each $A \in \targetset$ define $\climber^A: \naturals \to A$ by $\climber^A_n = \max \interceptor^A_n$.\\

\textbf{Example}\\
Let $A = \{ \elt \in \sourceset :\elt < 1 \}.$ Then  $\climber^A = \frac{1}{2}, \frac{3}{4},\frac{7}{8},\frac{15}{16},\frac{31}{32},\frac{63}{64},\frac{127}{128}, ... $\\

\textbf{Proposition}\\
$\forall n \in \naturals  \;\;f^A_n \le f^A_{n + 1}$.\\

(1) $\nondisjoints_{n+ \wed^A - 1} \subset \nondisjoints_{n+1 + \wed^A - 1} \implies \interceptor^A_n \subset \interceptor^A_{n+1} \implies  \max \interceptor^A_n \le \max \interceptor^A_{n+1} \implies f^A_n \le f^A_{n + 1}$.\\

 \textbf{Proposition}\\
 $\forall n \in \naturals \;\; f^A_n \in A.\\

\textbf{Proposition}\\
$\forall n \in \naturals \;\; f^A_n + 2^{-n} \notin A.\\

(1)$f^A_n +2^{-n} \in A \implies f^A_n + 2^{-\alpha^A - n + 1} \in A \implies \max \interceptor^A_n = f^A_n < f^A_n + 2^{-\alpha^A - n + 1} \le \max \interceptor^A_n$, a contradiction.\\

\textbf{Proposition}\\
$\forall a \in A \;\; \exists n \in \naturals \;\; a < f^A_n.$\\ 

(1) $a \in A \in \targetset \implies \exists a' \in \targetset \;\; a <  a'$.\\
(1) $a' \in A = \bigcup_{n = 0}^\infty \interceptor^A_n \implies \exists n_0 \in \naturals \;\; a' \in \interceptor^A_{n_0} \implies a < a' \le f^A_{n_0} = \max \interceptor^A_{n_0}$.\\

\section{A Helper Function}

\textbf{Definition}\\
For $A \in \targetset$ define $\leaper^A : A \to A$ by $\leaper^A(a) = \carpenter^A_{m_0}$, where $m_0 = \min \{ m \in \naturals : \carpenter^A(m) > a \}.$\\


\section{An Upper Bound}

\textbf{Definition}\\
For $A \in \targetset$ define $\umbrella^A= \min \{ k \in \sourceset : k \notin A \}.$\\ 

\section{Multiplication}\\

\textbf{Definition}\\
For $A, B \in \targetset$ define $AB = \{ \elt \in \sourceset : \exists a' \in A \;\; \exists b' \in B \;\; \elt < a'b' \}.$\\

\textbf{Proposition}\\
$A, B \in \targetset \implies AB \in \targetset$.\\ 

(1) $2^{-m_0} < \carpenter^A_1 \carpenter^B_1  \implies 2^{-m_0} \in AB$.\\
(2) $\umbrella^A \umbrella^B \in AB \implies \exists a \in A \;\; \exists b \in B \;\; \umbrella^A \umbrella^B < ab < \umbrella^A \umbrella^B$, a contradiction. \\
(3) $ \altelt < \elt' \in AB \implies \;\; \exists a \in A \;\; \exists b \in B \;\; \altelt < \elt' < ab \implies \altelt \in AB$.\\ 
(4) $\altelt \in AB \implies \exists a \in A \;\; \exists b \in B \;\; \altelt < ab  < \leaper^A(a) \leaper^B(b) \implies ab \in AB.$\\


\section{A Multiplicative Identity}\\



\textbf{Definition}\\
Define $I = \{\elt \in \sourceset : \elt < 1 \}$.\\

\textbf{Proposition}\\
$\forall A \in \targetset \;\; AI = A$.\\

(1) $\elt \in AI \implies \elt < a'i < a' \implies \elt \in A$. Hence $AI \subset A$.\\
(2) $\elt \in A \implies \elt < \leaper^A(\elt) \in A \implies \exists n_0 \;\; \elt < \leaper^A(\elt)(1-2^{-n_0}) \implies a \in AI.$ Hence $A \subset AI$. \\

\section{A Simple Square Root}\\

\textbf{Definition}\\
Define $\rooter : \sourceset \to \targetset$ by $\rooter(\altelt) = \{\elt \in \sourceset: \elt^2 < \altelt\}.$\\ 

\textbf{Proposition}\\
$[f^{\rooter(p)}_n]^2 \nearrow p$.\\

(1) $[f^{\rooter(p)}_n]^2 < p < [ f^{\rooter(p)}_n + 2^{-n} ]^2 = [f^{\rooter(p)}_n]^2  + 2^{1-n} f^{\rooter(p)}_n +  2^{-2n} < p + 2^{1-n} \umbrella^{\rooter(p)} + 2^{-2n}$.\\
(2) $0 < p - [f^{\rooter(p)}_n]^2 < 2^{1-n} \umbrella^{\rooter(p)} + 2^{-2n} \to 0$.\\

\textbf{Proposition}\\
$[f^{\rooter(p)}_n +2^{-n}]^2 \searrow p$.\\

(1) $0 < [ f^{\rooter(p)}_n + 2^{-n} ]^2 - p < [ f^{\rooter(p)}_n^2 + 2^{-n} ]^2 -[f^{\rooter(p)}_n]^2 < 2^{1-n} \umbrella^{\rooter(p)} + 2^{-2n} \to 0$.\\

\textbf{Proposition}\\
$\forall \altelt \in \sourceset \;\; \rooter(\altelt) \in \targetset$.\\

(1) $ 2^{-n_0} < \min\{s,1\} \implies (2^{-n_0})^2 = 2^{-2n_0} < 2^{-n_0} < \altelt \implies 2^{-n_0} \in  \rooter(\altelt)$.\\
(2) $\elt = \max\{\altelt ,1\} \implies \elt^2 \ge \elt \ge \altelt \implies \elt \notin  \rooter(\altelt).$\\ 
(3) $\elt < \altelt \in  \rooter(\altelt) \implies \elt^2 < \altelt^2 < x \implies \elt \in \rooter(\altelt)$.\\
(4) $\elt \in \rooter(\altelt) \implies \elt^2 < \altelt \implies  (\elt + 2^{-n_0})^2 = \elt^2 + 2^{1-n_0}\elt + 2^{-2n_0} < \altelt \implies (\elt + 2^{-n_0})^2 \in   \rooter(\altelt)$.\\

\textbf{Proposition}\\
$\forall \altelt \in \sourceset \;\; \rooter(\altelt)\rooter(\altelt) = \homo(\altelt).\\ 

(1) $x \in \rooter(\altelt)\rooter(\altelt) \implies x < q_0q_1 < q^2_{\max} < \altelt \implies  x \in \homo(\altelt)$. Hence $\rooter(\altelt)\rooter(\altelt) \subset \homo(\altelt)$\\
(2) $p \in \homo(\altelt)$ and $[ f^{\rooter(p)}_n + 2^{-n} ]^2 \searrow p \implies \exists n_0 \in \naturals \;\;  p < [ f^{\rooter(p)}_{n_0} + 2^{-n_0} ]^2 < s$.\\
(3) $f^{\rooter(p)}_{n_0} + 2^{-n_0} \in \rooter(\altelt) \implies p \in \rooter(\altelt)\rooter(\altelt)$. Hence $\homo(\altelt) \subset \rooter(\altelt)\rooter(\altelt)$.\\ 


\section{Generalizing The Simple Square Root}\\

\textbf{Definition}\\
For all $A \subset \sourceset$, define $\sqrt{A} = \{\elt \in \sourceset:  \elt^2 \in A\}.$\\



\textbf{Proposition}\\
$\forall A \in \targetset \;\; \sqrt{A} \in \targetset$.\\

(1) $ 2^{-n_0} \in P \implies 2^{-2n_0} < 2^{-n_0} \in P \implies 2^{-2n_0} \in P \implies 2^{-n_0} \in \sqrt{P}.$\\
(2) $p <  \umbrella_P < \umbrella_P^2  \implies \umbrella_P \notin  \sqrt{P}$.\\ 
(3) $\elt < \altelt \in  \sqrt{P} \implies \elt^2 < \altelt^2 \in P \implies \elt^2 \in P \implies \elt \in \sqrt{P}.$\\ 
(4) $\elt \in \sqrt{P} \implies \elt^2 < p_0 \in P \implies  (\elt + 2^{-n_0})^2 < p_0 \implies \elt + 2^{-n_0} \in   \sqrt{P}$.\\


\textbf{Proposition}\\
$\forall A \in \targetset \;\; \sqrt{A} \sqrt{A} = A$.\\ 

(1) $\elt \in \sqrt{A} \sqrt{A}\implies \exists r_0, r_1 \in \sqrt{A} \;\;\; \elt < r_0r_1 \le r_{\max}^2 \in A \implies \elt \in A$. Hence $\sqrt{A} \sqrt{A} \subset A.$\\
(2) $a \in A$ and  $[f^{\rooter(a)}_n + 2^{-n}]^2 \searrow a \implies \exists n_0 \in \naturals \;\; a < [f^{\rooter(a)}_{n_0} + 2^{-n_0}]^2 < \leaper^A(a) $.\\
(3) $ f^{\rooter(a)}_{n_0} + 2^{-n_0} \in \sqrt{A} \implies a \in \sqrt{A}\sqrt{A}$. Hence $A \subset \sqrt{A} \sqrt{A}$.\\ 

\section{A Multiplicative Inverse}\\

\textbf{Definition}\\
For $A \subset \sourceset$, define $\invmark A := \{ \altelt \in \sourceset : \exists n \in \naturals \;\; \forall a \in A \; [\altelt + 2^{-n}]a < 1\}$.\\

\textbf{Proposition}\\
$\forall A \in \targetset \;\; \invmark A \in \targetset.$\\

(1) $[2^{-m_0} + 2^{-m_0} ]\umbrella^A < 1 \implies [2^{-m_0} + 2^{-m_0} ]a < [2^{-m_0} + 2^{-m_0} ]\umbrella^A < 1 \implies 2^{-m_0} \in \invmark A.$\\
(2) $2^{-n_0} < \carpenter^A_1 \implies 1 < 2^{n_0} \carpenter^A_1 \implies 2^{n_0} \notin \invmark A$.\\
(3) $y < x \in \invmark A \implies  \forall \elt \in A \;\; [ y + 2^{-m_0} ]a <  [ x + 2^{-m_0} ]a< 1 \implies y \in \invmark A$.\\
(4) $x \in \invmark A \implies \forall \elt \in A \;\; [x + 2^{-m_0}]a = [x + 2^{-m_0-1} + 2^{-m_0-1}]a < 1 \implies x + 2^{-m_0-1} \in \invmark A$.\\


\textbf{Definition}\\
Define $\prevert^{A}_n = \carpenter^A_{m_n}$ where $m_n = \min\ M_n$, where $M_n = \{ m \in \naturals : \carpenter^A_m [ f^{\invmark A}_n + 2^{-n}] \ge 1 \}$.\\

\textbf{Note}\\
The set $M_n$  above is not empty for any $n$, else $f^{\invmark A}_n + 2^{-n} \in \invmark A$, which is impossible by the definition of $f^{\invmark A}_n$.\\

\textbf{Proposition}\\
$\forall n \in \naturals \;\; 0 < f^{\invmark A}_n \prevert^{ A}_n < 1.$\\




\textbf{Proposition}\\
$\forall A \in \targetset \;\;  A \invmark A = I.$ \\

(1) $x \in A \invmark A \implies x < a_0 \invmark a_0 < 1 \implies x \in I$. Hence $A \invmark A \subset I$.\\  
(2) $ f^{\invmark A}_n \prevert^{A}_n < 1 \le [f^{\invmark A}_n +2^{1-n}] \prevert^{ A}_n = f^{\invmark A}_n \prevert^{ A}_n + 2^{1-n}\prevert^{A}_n < 1 + 2^{1-n}\umbrella^A \implies 0 < 1 - f^{\invmark A}_n \prevert^{ A}_n < 2^{1-n}\umbrella^A   \implies f^{\invmark A}_n \prevert^{ A}_n \to 1$.\\
(3) $ p \in I \implies 0 < p < 1 \implies \exists n_0 \in \naturals \;\; 0 < p <  f^{\invmark A}_{n_0}  \prevert^{ A}_{n_0} < 1 \implies p \in A \invmark A$. Hence $ I \subset A \invmark A $.\\ \\

\section{The Structure of $\targetset$}


\textbf{Definition}\\
For $A \in \targetset$, define $\Segmenter^A = \bigcup_{B \in \targetset}^ {B < A} B$.\\

\textbf{Note}\\
In other words, $\Segmenter^A$ is the union of all elements in $\targetset$ that are less than $A$.\\

\textbf{Proposition}\\
$\forall A \in \targetset \;\;\; A = \Segmenter^A $.\\


(1) $a  \in A \implies a  < \leaper^A(a ) \implies a  \in \homo(\leaper^A(a ) ) < A \implies a \in  \Segmenter^A$. Hence $A \subset \Segmenter^A$.\\
(2) $a \in \Segmenter^A \implies \exists A' \in \targetset \;\; a \in A' < A \implies  a \in A.$ Hence $\Segmenter^A \subset  A.$\\


\end{document}
