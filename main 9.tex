\documentclass{article}
\usepackage[utf8]{inputenc}

\documentclass{article}
\usepackage[utf8]{inputenc}
\usepackage[usenames]{color} %used for font color
\usepackage{amssymb} %maths
\usepackage{amsmath} %maths
\usepackage{setspace}
\usepackage{ mathrsfs }
\usepackage[margin=0.5in]{geometry}
\setlength{\parindent}{0pt}
\newcommand{\odds}{ \mathscr{O}}
\newcommand{\evens}{ \mathscr{E}}

\newcommand{\rationals}{ \mathscr{Q}}
\newcommand{\targetset}{ \mathscr{X}}
\newcommand{\sourceset}{ \mathscr{P}}
\newcommand{\reals}{ \mathbb{\elt}}

\newcommand{\disjoints}{ X}
\newcommand{\nondisjoints}{E}
\newcommand{\naturals}{ \mathscr{N}}
\newcommand{\carpenter}{ \omega }
\newcommand{\leaper}{l}
\newcommand{\climber}{ f}
\newcommand{\homo}{h }
\newcommand{\elt}{q}
\newcommand{\homoelt}{p}
\newcommand{\altelt}{p}
\newcommand{\rooter}{\Psi}
\newcommand{\invmark}{\overline}
\newcommand{\odds}{ \mathscr{O}}
\newcommand{\sequencer}{\sigma}
\newcommand{\umbrella}{\beta}
\newcommand{\prevert}{\zeta}
\newcommand{\targetname}{bundle }
\newcommand{\targetnameplural}{bundles }
\newcommand{\intersector}{\Omega}
\newcommand{\interceptor}{\Xi}


\title{Bundles}
\author{S. Z. }
\date{January 2022}

\begin{document}

\maketitle

\textbf{Note}\\
The following is a construction of (dyadic) Dedekind \textbf{rods} as opposed to the famous cuts. Only positive real-like numbers are developed, though it's easy from here to use equivalence classes of pairs of rods to get the rest.  Instead of thinking of the real number as a parting of the rationals, we can also think of each real number as a measuring rod. In this case each rod is a well-ordered and countably infinite set of simpler dyadic rods. The magnitude suggested by such a infinite set is the smallest stretch of space in which all of the simple rods will fit. This presentation takes some experience with limits for granted. Only the more difficult propositions are proved, and these are proved somewhat informally.\\


\textbf{Definition}\\
Let $\naturals $ be the set of positive integers.\\

\textbf{Definition}\\
For $n \in \naturals$, define  $\nondisjoints_n  = \{j/n : j \in \naturals \}$. \\


\textbf{Definition}\\
Define $\sourceset = \bigcup_{n = 1}^\infty \nondisjoints_n$.\\

\textbf{Note}\\
This means that $\sourceset$ is just the set of positive rational numbers.\\

\textbf{Definition}\\
For $A,B \subset \sourceset$, define $A < B$ to mean $A \subsetneqq B $.\\

\textbf{Definition}\\
Define $\targetset = \{A \subset \sourceset : \;\; \emptyset < A < \sourceset,\;\; \altelt < \elt \in A \implies \altelt \in A,\;\; \forall \elt \in A \;\; \exists \altelt  \in A \;\; \elt < \altelt \}$.\\

\textbf{Note}\\
The entities defined above are nonempty, proper subsets of the positive rationals which are closed downward and have no maximum.\\

\section{Injecting $\sourceset$ into $\targetset$}\\

\textbf{Definition}\\
For $\altelt \in \sourceset$, define $(\elt < \altelt) = \{\elt \in \sourceset : \elt < \altelt \}$.\\

\textbf{Definition}\\
Define $\homo : \sourceset \to \targetset$ by $\homo(\altelt)  = (\elt < \altelt)}$.\\

\textbf{Proposition}\\
$\homoelt \in \sourceset \implies \homo(\homoelt) \in \targetset.$\\

(1) $0 < \homoelt / 2 < \homoelt \implies \homoelt/2 \in \homo(\homoelt) \implies \emptyset < \homo(\homoelt)$.\\ 
(2) $\elt \in \homo(\altelt) \implies \elt < \altelt \implies \altelt \notin\homo(\homoelt) \implies \homo(\homoelt) < \sourceset$.\\
(3) $\elt < \elt' \in \homo(\altelt) \implies 0 < \elt < \elt' < \homoelt \implies \elt \in \homo(\homoelt).$\\ 
(4) $\elt \in \homo(\homoelt) \implies \elt < \homoelt \implies \elt < (\elt+ \homoelt)/2 < \homoelt \implies (\elt+ \homoelt)/2 \in \homo(\homoelt)$. \\

\textbf{Proposition}\\
$\forall \elt,\elt' \in \sourceset \;\; \homo(\elt) = \homo(\elt') \implies \elt = \elt'$.\\

\section{Ordering each element of \targetset$}\\


\textbf{Definition}\\
For $A \in \targetset$ and $ n \in \naturals$, define $\intersector^A_n = A \cap \nondisjoints_n$.\\





\textbf{Proposition}\\
$\forall n \in \naturals \;\; |\intersector^A_n| \in \naturals$.\\

\textbf{Proposition}\\
$A = \bigcup_n^\infty \intersector^A_n$.\\

\textbf{Definition}\\
Let $a \in \intersector^A_i, a' \in \intersector^A_j$ and define $a \prec a'$ if (1) $i < j$ or (2) if $i = j$ and $a <  a'$.\\



\textbf{Proposition}\\
The order $\prec$ can be used to enumerate A, with elements appearing more than once.\\

\textbf{Definition}\\
For $A \in \targetset$, define $\carpenter^A : \naturals \to A$ to be sequence of $A$'s elements as ordered by $\prec.$\\

\textbf{Example}\\
Let $A = \{ \elt \in \sourceset : \elt < 1 \}.$ Then $\carpenter^A = \frac{1}{2},\frac{1}{3},\frac{2}{3},\frac{1}{4},\frac{2}{4},\frac{3}{4}, ... $ and $\carpenter^A_2 = \frac{1}{3}$.\\

\textbf{Definition}\\
Define $\mu^A = \min \{ m \in \naturals : 1/m \in A \}$. \\


\textbf{Definition}\\
For $A \in \targetset$ and $ n \in \naturals$, define $\interceptor^A_n = \intersector^A_{n + \mu^A - 1}.$\\ 


\textbf{Proposition}\\
$\forall n \in \naturals \;\; \interceptor^A_n \ne \emptyset$.\\
 
\textbf{Definition}\\
For each $A \in \targetset$, define $\climber^A_n = \max  \interceptor^A_n$.\\

\textbf{Example}\\
Let $A = \{ \elt \in \sourceset :\elt < 1 \}.$ Then $\mu^A = 1$ and $\climber^A = \frac{1}{2}, \frac{3}{4},\frac{7}{8},\frac{15}{16},\frac{31}{32},\frac{63}{64},\frac{127}{128}, ... $.$\\

\textbf{Proposition}\\
$\forall n \in \naturals \;\; \forall k \in \naturals \;\;f^A_n \le f^A_{kn}$.\\

(1) $\interceptor^A_n \subset \interceptor^A_{kn} \implies \max \interceptor^A_n \le \max \interceptor^A_{kn} $.\\
 
 \textbf{Proposition}\\
 $\forall n \in \naturals \;\; f^A_n \in A.\\

\textbf{Proposition}\\
$\forall n \in \naturals \;\; f^A_n + 2^{-n} \notin A.\\

(1)$f^A_n = \max A \cap \nondisjoints_{\mu^A + n} <  f^A_n + 2^{-\mu^A - n} \in \nondisjoints_{\mu^A + n} \implies \climber^A_n + 2^{-\mu^A - n} \notin A \implies \climber^A_n + 2^{- n} \notin A.$\\

\textbf{Proposition}\\
$\forall a \in A \;\; \exists n \in \naturals \;\; a \le f^A_n.$\\ 

(1) $a \in A = \bigcup (A \cap \nondisjoints_n) \implies a \in A \cap \nondisjoints_{n_0} \implies a \le f^A_{n_0} = \max A \cap \nondisjoints_{n_0}$\\

\textbf{Definition}\\
For $A \in \targetset$ define $\leaper^A : A \to A$ by $\leaper^A(a) = \carpenter^A_{m_0}$, where $m_0 = \min \{ m \in \naturals : \carpenter^A(m) > a \}.$\\

\textbf{Definition}\\
For $A \in \targetset$ define $\umbrella^A= \min \{ k \in \sourceset : k \notin A \}.$\\ 

\textbf{Definition}\\
For $A, B \in \targetset$ define $AB = \{ \elt \in \sourceset : \exists a' \in A \;\; \exists b' \in B \;\; \elt < a'b' \}.$\\

\textbf{Proposition}\\
$A, B \in \targetset \implies AB \in \targetset$.\\ 

(1) $2^{-m_0} < \carpenter^A_1 \carpenter^B_1  \implies 2^{-m_0} \in AB$.\\
(2) $ \altelt < \elt' \in AB \implies \altelt < \elt' < ab \implies \altelt \in AB$.\\
(3) $\elt \in AB \implies \elt < a'b'  < \leaper^A(a') \leaper^B(b') \implies a'b' \in AB.$\\
(4) $\umbrella^A \umbrella^B \in AB \implies \umbrella^A \umbrella^B < a'b' < \umbrella^A \umbrella^B$, hence $\umbrella^A \umbrella^B \notin AB.$\\ 

\textbf{Definition}\\
Define $I = \{\elt \in \sourceset : \elt < 1 \}$.\\

\textbf{Proposition}\\
$\forall A \in \targetset \;\; AI = A$.\\

(1) $\elt \in AI \implies \elt < ai < a \implies \elt \in A$. Hence $AI \subset A$.\\
(2) $\elt \in A \implies \elt < \leaper^A(\elt) \in A \implies \exists n_0 \;\; \elt < \leaper^A(\elt)(1-2^{-n_0}) \implies a \in AI.$ Hence $A \subset AI$. \\


\textbf{Definition}\\
Define $\rooter : \sourceset \to \targetset$ by $\rooter(\altelt) = \{\elt \in \sourceset: \elt^2 < \altelt\}.$\\ 

\textbf{Proposition}\\
$[f^{\rooter(p)}_n]^2 \nearrow p$.\\

(1) $[f^{\rooter(p)}_n]^2 < p < [ f^{\rooter(p)}_n + 2^{-n} ]^2 = [f^{\rooter(p)}_n]^2  + 2^{1-n} f^{\rooter(p)}_n +  2^{-2n} < p + 2^{1-n} \umbrella^{\rooter(p)} + 2^{-2n}$.\\
(2) $0 < p - [f^{\rooter(p)}_n]^2 < 2^{1-n} \umbrella^{\rooter(p)} + 2^{-2n} \to 0$.\\

\textbf{Proposition}\\
$[f^{\rooter(p)}_n +2^{-n}]^2 \searrow p$.\\

(1) $0 < [ f^{\rooter(p)}_n + 2^{-n} ]^2 - p < [ f^{\rooter(p)}_n^2 + 2^{-n} ]^2 -[f^{\rooter(p)}_n]^2 < 2^{1-n} \umbrella^{\rooter(p)} + 2^{-2n} \to 0$.\\

\textbf{Proposition}\\
$\rooter(\sourceset) \subset \targetset$.\\

(1) $ 2^{-n_0} < \min\{x,1\} \implies (2^{-n_0})^2 = 2^{-2n_0} < 2^{-n_0} < \altelt \implies 2^{-n_0} \in $ \rooter(\altelt)$. 

(2) $\elt < \altelt \in  \rooter(\altelt) \implies \elt^2 < \altelt^2 < x \implies \elt \in \rooter(\altelt)$.

(3) $\elt \in \rooter(\altelt) \implies \elt^2 < \altelt \implies  (\elt + 2^{-n_0})^2 = \elt^2 + 2^{1-n_0}\elt + 2^{-2n_0} < \altelt \implies (\elt + 2^{-n_0})^2 \in $  \rooter(\altelt)$.

(4) $\elt = \max\{\altelt ,1\} \implies \elt^2 \ge \elt \ge \altelt \implies \elt \notin $ \rooter(\altelt).$\\ 

\textbf{Proposition}\\
$\forall \altelt \in \sourceset \;\; \rooter(\altelt)\rooter(\altelt) = \homo(\altelt).\\ 

(1) $x \in \rooter(\altelt)\rooter(\altelt) \implies x < q_0q_1 < q^2_{\max} < \altelt \implies  x \in \homo(\altelt)$. Hence $\rooter(\altelt)\rooter(\altelt) \subset \homo(\altelt)$\\
(2) $p \in \homo(\altelt)$ and $[ f^{\rooter(p)}_n + 2^{-n} ]^2 \searrow p \implies \exists n_0 \in \naturals \;\;  p < [ f^{\rooter(p)}_{n_0} + 2^{-n_0} ]^2 < s$.\\
(3) $f^{\rooter(p)}_{n_0} + 2^{-n_0} \in \rooter(\altelt) \implies p \in \rooter(\altelt)\rooter(\altelt)$. Hence $\homo(\altelt) \subset \rooter(\altelt)\rooter(\altelt)$.\\ 

\textbf{Definition}\\
For all $A \subset \sourceset$, define $\sqrt{A} = \{\elt \in \sourceset:  \elt^2 \in A\}.$\\



\textbf{Proposition}\\
$A \in \targetset \implies \sqrt{A} \in \targetset$.\\

(1) $ 2^{-n_0} \in P \implies 2^{-2n_0} < 2^{-n_0} \in P \implies 2^{-2n_0} \in P \implies 2^{-n_0} \in \sqrt{P}.$\\
(2) $\elt < \altelt \in  \sqrt{P} \implies \elt^2 < \altelt^2 \in P \implies \elt^2 \in P \implies \elt \in \sqrt{P}.$\\ 
(3) $\elt \in \sqrt{P} \implies \elt^2 < p_0 \in P \implies  (\elt + 2^{-n_0})^2 < p_0 \implies \elt + 2^{-n_0} \in   \sqrt{P}$.\\
(4) $p <  \umbrella_P < \umbrella_P^2  \implies \umbrella_P \notin  \sqrt{P}$.\\ 

\textbf{Proposition}\\
$A \in \targetset \implies \sqrt{A} \sqrt{A} = A$.\\ 

(1) $\elt \in \sqrt{A} \sqrt{A}\implies \exists r_0, r_1 \in \sqrt{A} \;\;\; \elt < r_0r_1 \le r_{\max}^2 \in A \implies \elt \in A$. Hence $\sqrt{A} \sqrt{A} \subset A.$\\
(2) $a \in A$ and  $[f^{\rooter(a)}_n + 2^{-n}]^2 \searrow a \implies \exists n_0 \in \naturals \;\; a < [f^{\rooter(a)}_{n_0} + 2^{-n_0}]^2 < \leaper^A(a) $.\\
(3) $ f^{\rooter(a)}_{n_0} + 2^{-n_0} \in \sqrt{A} \implies a \in \sqrt{A}\sqrt{A}$. Hence $A \subset \sqrt{A} \sqrt{A}$.\\ 

\textbf{Definition}\\
For $A \subset \sourceset$, define $\invmark A := \{ \elt \in \sourceset : \exists n \in \naturals \;\; \forall a \in A \; [\elt + 2^{-n}]a < 1\}$.\\

\textbf{Proposition}\\
$A \in \targetset \implies \invmark A \in \targetset.$\\

(1) $[2^{-m_0} + 2^{-m_0} ]\umbrella^A < 1 \implies [2^{-m_0} + 2^{-m_0} ]a < [2^{-m_0} + 2^{-m_0} ]\umbrella^A < 1 \implies 2^{-m_0} \in \invmark A.$\\
(2) $y < x \in \invmark A \implies  \forall \elt \in A \;\; [ y + 2^{-m_0} ]a <  [ x + 2^{-m_0} ]a< 1 \implies y \in \invmark A$.\\
(3) $x \in \invmark A \implies \forall \elt \in A \;\; [x + 2^{-m_0}]a = [x + 2^{-m_0-1} + 2^{-m_0-1}]a < 1 \implies x + 2^{-m_0-1} \in \invmark A$.\\
(4) $2^{-n_0} < \carpenter^A_1 \implies 1 < 2^{n_0} \carpenter^A_1 \implies 2^{n_0} \notin \invmark A$.\\

\textbf{Definition}\\
Define $\prevert^{A}_n = \carpenter^A_{m_n}$ where $m_n = \min\ M_n$, where $M_n = \{ m \in \naturals : \carpenter^A_m [ f^{\invmark A}_n + 2^{-n}] \ge 1 \}$.\\

\textbf{Note}\\
The set $M_n$  above is not empty for any $n$, else $f^{\invmark A}_n + 2^{-n} \in \invmark A$, which is impossible by the definition of $f^{\invmark A}_n$.\\

\textbf{Proposition}\\
$\forall n \in \naturals \;\; 0 < f^{\invmark A}_n \prevert^{ A}_n < 1.$\\




\textbf{Proposition}\\
$A \in \targetset \implies A \invmark A = I.$ \\

(1) $x \in A \invmark A \implies x < a_0 \invmark a_0 < 1 \implies x \in I$. Hence $A \invmark A \subset I$.\\  
(2) $ f^{\invmark A}_n \prevert^{A}_n < 1 \le [f^{\invmark A}_n +2^{1-n}] \prevert^{ A}_n = f^{\invmark A}_n \prevert^{ A}_n + 2^{1-n}\prevert^{A}_n < 1 + 2^{1-n}\umbrella^A \implies 0 < 1 - f^{\invmark A}_n \prevert^{ A}_n < 2^{1-n}\umbrella^A   \implies f^{\invmark A}_n \prevert^{ A}_n \to 1$.\\
(3) $ p \in I \implies 0 < p < 1 \implies \exists n_0 \in \naturals \;\; 0 < p <  f^{\invmark A}_{n_0}  \prevert^{ A}_{n_0} < 1 \implies p \in A \invmark A$. Hence $ I \subset A \invmark A $.\\ \\


\textbf{Definition}\\
For $A \subset \sourceset$ and $\altelt \in \sourceset$ define $A < \elt$ if $\forall a \in A \;\; a < \altelt$. \\

\textbf{Definition}\\
For $A,A' \in \targetset$, define $A < A'$ iff $ A'- A \ne \emptyset$.\\

\textbf{Proposition}\\
$[\;\; \exists B \in \targetset \;\; \forall n \in \naturals \;\; B > A_n \in \targetset \;\;] \implies \bigcup_{n=1}^\infty A_n \in \targetset.$\\

\textbf{Proposition}\\
$[\;A \in \targetset \; \wedge \;\; \forall n \in \naturals \;\; A_n = \homo(\carpenter^A_n) \;] \implies  \bigcup_{n=1}^\infty A_n =A.$\\

(1) $a \in A \implies a < \leaper^A(a) = \carpenter^A_{n_0} \implies a \in A_{n_0}   \implies a \in \bigcup_{n=1}^\infty A_n$. Hence $A \subset \bigcup_{n=1}^\infty A_n.$\\
(2) $\altelt \in A_n \implies \altelt < \carpenter^A_n \in A \implies \altelt \in A.$ Hence \bigcup_{n=1}^\infty A_n \subset A$.\\


\end{document}


\section{Introduction}

\end{document}
