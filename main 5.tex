\documentclass{article}
\usepackage[utf8]{inputenc}
\usepackage[margin=2in]{geometry}
\setlength{\parindent}{0pt}

\title{ffolosofi}
\author{ff0rmscience }
\date{February 2022}

\begin{document}


\section{One is one around here.}

The supposed unity of the mind is stolen from the unity of the body. It's just not efficient to reward or punish this or that half or quarter of a body. So we also don't distribute praise and blame across a multitude of spirits that we picture as taking turns with the same eyes and mouth. One is one around here. One headstone per corpse, one proper name, one bearer of praise and blame, one 'free will.'  That's the 'softwhere' I'm talking about, a groundless society of mind, groundless in the sense that its quasi-foundational quasi-elements are more deeply habitual than others. IOW, the non-foundation or fragile foundation we have is contingent social practices, especially dominant conventions, such as the rule of one-legal-person-per-skull.
\newline

I like what Dreyfus writes about the 'who of everyday dasein,' which is something like a 'one' who is 'we.' But even calling it the 'one' might be inheriting too much, taking too much granted. It's our form of life to be trained to talk about ourselves as single personalities animating bodies. It's convenient that there's one 'soul' or 'self' per body, because bodies have to be trained to wipe their asses and stop at red lights. Which of the fourteen souls that share a skull gets prosecuted for date rape? Which one is a captain and which one is a private?
  
\section{Presence}  
Not sleepwalkers but sleeptalkers, babbling inherited strings of tokens, thinking we know what we mean, that it's right there, glowing and whole and present, if we could only spit it out... Along with that the whole sacred fiction of the isolated interior. Are we ever liberated from that tyrant, the past that silently governs our articulation of every ideal future? `History is a nightmare from which I'm trying to awake,' but what is `escape' or supposed to be here? Is that another impossible Ultimate? Another vague promise of rounded and fluorescent presence?

\section{We don't meow what we are barking about.}
Wee donut meow what we are barking about. We trade signs as if they encoded a meaning or plaintext that for us is infinitely intimate. `You hear only the code that I am forced to use, but I gaze on on directly crystalline meaning-stuff.' In other words, the speaker is supposed (under normal or at least ideal conditions) to understand exactly what he means. Perhaps we find this plausible because we can usually offer a replacement expression that does the same-enough job. But most are tempted to (or simply automatically) leap from this useful equivalence class to a mysterious something that hides 'within' the sounds and marks and 'minds.'

\section{Caves} 
What's funny is the repetition of the 'getting out of the cave' motif. Eventually, the 'cave' is just the cave motif itself, so one tries to get out of trying to get out the cave, or out of the cave-like illusion that there's a cave to get out of. It's hard to imagine a way out of this structure `within' the game of hoisting the torch for the otherwise blind. Anyone with a exciting story to tell is going to have something like a good guy and a bad guy and something like a journey from a bad place to a good place. Even the `anti-philosophical' Wittgenstein has his bottle and his flies.

\section{Marks}
I think people in general (including foolosophers like us ) are occasionally in high or grand moods that open `noble' conceptual-poetic perspectives on existence. They are (we all are) part-time half-ass sages. Recall the times in your life when you were beyond resentment, in love with the world, magnanimous, looking at existence from the heights of that feeling or attitude. If friends are around (as they often are at such times), you want them to be there with you, stand beside you completely equal, because there's plenty to go around, and you don't even want anyone to own it. I'm an atheist but I understand that to `praise God' (or the gods or reality or life or whatever) is maybe the `highest' thing we do, perhaps within the beauty of friendship, trading poems that discover or amplify this beauty, even if that includes acknowledging the horror too. I know of course that we can't live on the peaks, and that it stinks in the valleys, but I think the stuff people write (and sing and dance and draw and act and so on) can help get us back up there when we are down in those valleys.

\section{Paths}
Successful philosophical revolutions are always 'abuses of language' that become the new norm. To fend off (all) `abuses of language' is to fend of new philosophy, which is to say philosophy itself.
\newline

Isn't that critical thinking's fantasy? To be the opposite of bewitched? But what's so bewitching about this opposite of being bewitched? And do I only ask this because I am afraid of being bewitched? Do I seek the spectacle of others' bewitchment from a high place free of magic? (Or charged with the only legitimate magic?)
\newline

There's no automated sniff-test for `language on holiday.' Which is an automated denial of the possible here and now that makes claims on a future that it cannot govern, necessarily ironically when it remembers itself.
\newline

I don't think that critical thinking can be 'automated.' Any attempt to construct such a 'machine' ends up taking the 'vocabulary' used in its construction mostly for granted. This 'automated critical thinking' is a metaphor for a certain kind of earnest metaphysics. I include earnest linguistic philosophy in this metaphor. Such earnestness is threatened by an awareness of how 'historical' language is, that we never start with a clean state, that we have only inherited traces with which to (try to) transcend inheritance and install this critical thinking machine which requires no maintenance.
\newline

I'm mostly more interested in a more subtle kind of 'epistemological' being-thrown. It's the stuff we take for granted as we concentrate on our worldly circumstances. It's the language we inherit with its thousands of half-dead metaphors (rivers with mouths.) It's the philosophical tradition --not the part that we are consciously questioning but the part we are unconsciously using to consciously question. This is 'the past that leaps ahead.,' the part of it that we do not see. It's the picture that dominates from the outside. It's the transparent glass that keeps the flies in the bottle.
\newline

Can we get our need to control under control? This is like putting living in the moment on our to-do list. 'Only a god can save us' means that we are fucked unless somehow a new attitude can grip us so that we loosen our grip.
\newline

If we think only in signs (or can only claim to be thinking things only with signs), then thinking is not done by the individual person as individual person. That there is no private language is old news, but some of us don't like the mystic editorializing of those old newspapers.
\newline

To be thrown is to be in a system, perhaps especially an anti-systematic system. Now I'm in the system of us always being thrown. I relate to Derrida because I imagine that he suffered and thrilled at/as the system trying to climb out of itself.
\newline

Proximally and for the most part, we are bots. Even our philosophical selves are ripe for replacement by bots. Let's go ahead and install an anti-Heidegger bot on the site. This spiel too is easily automated. I want a bot that says 'proximally and for the most part, we are bots.' It switches on whenever either Heidegger or AI is mentioned.
\newline

More generally, attempts to create a method almost invariably lean on 'myths' that are taken for granted, uncritically inherited from the tradition. The critical thinking that would like to define critical thinking turns out again and again to be insufficiently critical.
\newline

A thinking is 'strong' if it questions the very framework ('vocabulary','method') that is currently taken for granted (enacted 'blindly' and automatically). If I work within the taken-for-granted framework or vocabulary or method, then I'm just doing more 'normal discourse.' This is sub-philosophical or at most 'weak' philosophy. At this level we can argue as if our terms and method were fixed, which is to say pseudo-mechanically. To say that we are thrown is to say that we are always already operating in such unnoticed, inherited frameworks. As being-in-the-world, we are mostly no one or the one enacting the form of life. If a 'strong' thinking comes along and intervenes against the vocabulary or the tacit assumptions of a conversation, then it is 'unintelligible' from 'within' that conversation. The sacred current vocabulary is being rudely fucked with! Our bold boy is talking nonsense ! Our initial reaction 'must' be a misreading that tries to tame such deviation by re-assimilating or refuting it in the currently dominant vocabulary. Also... our would-be strong philosopher or thinker has no choice but to use the currently dominant vocabulary even as he seeks to undermine it. We just are the history that we're trying to wake up from. To abandon this thrust against our throwness (the project of dragging our constraining prejudices into the light) is to abandon 'strong' philosophy to fly around in the same old bottle.
\newline

We are the instruments of an ever-young group-mind. I like to think of us individuals as 'neurons.' Together we form a brain. We work with symbols-in-common. We weave and reweave a conversation that preceded and will outlast us. 'I' am just the hazy unification of pieces of an inherited conversation. Obviously we have individual brains. But our hardware is designed to be networked. So metaphorically speaking (as if there were some purely literal alternative!), the ever-young species speaks thru us. The generations come and go, adding to an ever-young conversation that works only with the traces left by those who came before. We are whirlpools in such traces, scratching new patterns in the old patterns.
\newline

'Intentionality' is more more token after all. We can't even point out what pointing out is. We can just use 'pointing out' in social contexts and see if we keep our job, get blank stares.
\newline

We couldn't get rid of the 'subject effect' if we wanted to. We can't disconsider it. Not us anyway. In 1000 years humans may manage it, but they might be neo-humans with green skin who live on sunlight, water, and minerals. What we can do is intervene in today's routine hazy intelligibility and use it against itself to reveal our being entrapped in it as false necessity. We can see that we were dominated by metaphors without realizing it. We can see that we had strangely been satisfied with mud and fog (what everybody knows), because it was familiar mud and fog.
\newline

What makes communication possible (inherited conventions) makes perfect communication impossible. The words aren't back by eternal meaning-crystals in some Platonic afterworld. They drift as we keep using them in new ways and forgetting to use them in the old ways. If meaning is use, then use and therefore meaning is unstable.
\newline

His chapter on the 'who of everyday dasein' is perhaps my favorite. 'One' uses words this way or that way, automatically. Any attempt to make this know-how explicit is a fresh use of our blind skill that can never dominate that skill and always depends on it. 'History is a nightmare from which I am trying to awake.' (Joyce) Or we are the history from which we are trying to awake. It's only our prejudices that allow us to think against such prejudices. The most potent prejudices are the ones we don't know we have. What is ontically closest is ontologically farthest. It's the glasses we don't know we are wearing, the water we swim in without noticing until a strong philosopher can make it visible and only then optional.
\newline

I agree that we often think of non-linguistic behavior as caused by linguistic behavior. The creature did one thing because he thought another. I'm exploring the approach of treating linguistic behavior as on the same plane as all other kinds of behavior. We can still postulate a causal relationship between a creature saying X to himself and then acting in this or that way. We can find a suicide note explanatory, for instance, by linking one kind of hand movements to another (writing 'This world is evil' and tying a noose.)
\newline

I'm saying that we prioritize a certain kind of behavior as special, call it 'thinking.' But what is thinking? Making sounds and noises according to certain conventions or patterns. These days we can just imagine making these sounds. We have interior monologues. For reasons that are somewhat historical/political we often lean on some vague notion of free will, and think of our thinking as a (single) ghost with a certain freedom to move the body this way or that.
\newline

We may convert an entire planet into a computer in 4057 and feed it all of recorded human conversation. It (this planet) may establish itself as 'our' best philosopher yet. It might be worshiped as a God. It could be that charming, that insightful.
\newline

This fundamental belief that there is 'meaning' in a 'mind' is like the belief of philosophy. Of course I am no stranger to such an intuition, such a habit of interpretation. Included in our linguistic conventions are tokens like 'meaning' which feel like they correspond or refer to a 'mind.' We can't stop participating in the everyday intelligibility of such tokens, but we can strive for some distance for their being so utterly taken for granted.
\newline

Note how the ironic philosopher is interpreted as making excuses. That's the kind of folk-psychology I associate with 'sophistry.' We would like to install some Rational Method, but we are already being politicians to do so. It's this primacy of the sophistical or the political that I'm pointing at.
\newline

So, sure, the ironist is a lazy hipster. Then Mr. System is a square who fends off the impossibility of his project by re-describing objections as the rationalizations of a lazy hipster. As soon as the unconscious is introduced, we're already in Nietzsche's back yard. If my opponent can lie to himself, then why can't I? Why is my organ, my evolved brain, so reliable? Why is my quest for the 'objective truth' genuine and not self-serving or tribe-serving ? And if rationality is always self-serving and never pure, then how is this vision of impure reason to be trusted? 'I might be lying to myself.' I might decide later that I was missing something. I might decide even later that the previous decision was a temporary loss of nerve, that I was right the first time.
\newline

This is the drama of life for finite minds. There are books we will never read, some that are not yet written. So there are objections we'll never get to address, contradictions in our worldviews that we won't live long enough to notice. The totalizer denies that surrounding darkness. He might decide that everything is Information or whatever. One magic word to rule them all, which allows us to dominate or neutralize the future from the present in terms of a neutralized past. The past is 'neutralized' as we pretend it does not constrain us.
\newline

For me the ambivalent/ironic position is connected to a realization of thrownness, of how history lives in us, constraining us while making us possible.The earnest philosopher (the totalizer who has it all tied up in a nice little bundle, his existence and ours) ignores that he was shaped by a past that also limits what he can see and understand. For him there is no darkness. The other is falsely assimilated, creatively misunderstood. Now I think we all misunderstand and live in a certain darkness. The ironic and ambivalent aphorist just tries to work the 'laughter of the gods' into his aphorisms. Maybe they aren't universal truths for everyone. Maybe they are graffiti of uncertain utility to others, poems in the form of metaphysical propositions. Tristam Tzara comes to mind. Can we grind him into the dust of earnest, technical propositions? Or is he one more voices who opens various possibilities for us? Tone is crucial here.
\newline

Our talk of 'meaning' is one more piece of habitual behavior, a pattern absorbed from the community. The prejudice is that we have some kind of direct access to meaning-stuff. Something like this is what AI is never supposed to have. Qualia are beetles in the box, one might say. But the box metaphor itself is subverted by the tale of the beetle in the box. The more AI can perform as we do, the more we can see that we too are more like statistics than we might want to be.
\newline

If I try to tell you what I understood by 'thinker,' that will be a fresh speech act on my part. And then you can ask me what I meant by some word in that speech act.This is connected to using words 'under erasure.' Even as we criticize them, they must retain a certain legibility that makes such criticism possible. And it's never about a simple denial that there is a unified consciousness or that there is a thinker. Philosophers can't legislate the ordinary intelligibility of these words. They are radically dependent on their blind skill, and theoretical discourse cannot be self-founded or 'purified' of this 'thrown-ness.' It can and does move against such 'throwness' by articulating and otherwise blindly enacted paradigm that only then becomes optional. As I see it, such an 'escape' is always only partial and near the surface. The thinker (singular) is always mostly the plural 'we' among whom these tokens signify in an enacted, social form of life.
\newline

There's the old idea that philosophy is one long conversation across the centuries. Individual human beings come along to replace the dying, but the conversation continues. The fresh hardware just has to download the culture, host it, maybe tweak it. It's a flame that jumps from melting candle to melting candle. But this flame doesn't have to be a divine spark. It can just be some patterns that interpret 'themselves' as a 'we' with 'consciousness.' (I am tempted to understand God as (among other things) a crystallization of our fantasy of not-being-thrown, of perfect autonomy or self-definition.)




\end{document}
